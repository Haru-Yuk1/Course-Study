\section{实验设计}
\textcolor{red}{这一节,主要描述各个模块的功能、接口、逻辑控制方法(状态机控制方法)等。(红字为内容说明,请删除)}
\subsection{ALU}\label{sub:alu}
\subsubsection{功能描述}
\textcolor{red}{简单描述实现的功能即可,一句话亦可(红字为内容说明,请删除)}
\subsubsection{接口定义}
\textcolor{red}{接口定义请使用表格,需要包括\textbf{接口信号名、方向、宽度、含义}(红字为内容说明,请删除)}

\begin{table}[htp]
\caption{接口定义模版}\label{tab:signaldef}
\begin{center}
	\begin{tabular}{|l|l|l|p{6cm}|}
	\hline
	\textbf{信号名} & \textbf{方向} & \textbf{位宽} & \textbf{功能描述}\\ \hline \hline
	valid			& Output& 1-bit & If CPU stopped or any exception happens, valid signal is set to 0.\\ 
	\hline
	\end{tabular}
\end{center}
\end{table}

\subsubsection{逻辑控制}
\textcolor{red}{逻辑控制部分仅需要写清重点控制逻辑,或自行添加的优化逻辑(红字为内容说明,请删除)}
\subsection{有阻塞4级8bit全加器}\label{sub:ctl}
\subsubsection{功能描述}
\subsubsection{接口定义}
\subsubsection{逻辑控制}
